\documentstyle[11pt]{article}
\topmargin 0.0in
\headheight 0.0in
\headsep 0.0in
\textwidth 7.5in
\oddsidemargin -0.5in
\textheight 9in
\newcount\hh
\newcount\mm

% TIME OF DAY
\mm=\time
\hh=\time
\divide\hh by 60
\divide\mm by 60
\multiply\mm by 60
\mm=-\mm
\advance\mm by \time
\def\hhmm{\number\hh:\ifnum\mm<10{}0\fi\number\mm}

\begin{document}
\title{(Operation-less) INTERFACE FOR SORTING FUNCTIONALITY \\
       Supplemental to the \\
       Supply Capability Engine (SCE) Data Model Release 2.0}
\author{Tom Ervolina (ERVOLINA at YKTVMV), 
Robin Lougee-Heimer (RLH at YKTVMV) \and Dan Connors  (DPC at YKTVMV)}
\date{March 14, 1997\footnote[2]{This document was 
formatted on \today\ at \hhmm.}}
\maketitle
The data model for SCE Release 2.0 enables very general modeling of
co-product production through the introduction of {\em operations}. A
special case of co-product production is {\em sorting}. By sorting we
mean the case where a single (untested) part number is sorted into a
variety of (tested) part numbers. Due to the one-to-one relationship
between part numbers to be sorted and sorting operations, the SCE
operation input files for a sorting model can be automatically
generate from input information which does not explicitly mention
{\em operations}. The benefit of this is that it reduces the complexity of
the input data.
 
This document describes the (operation-free) user input files for
sorting and how the necessary SCE release 2.0 input files can be
generated from them. This document assumes the reader is familiar with
the data model for release 2.0.  The following items are
important:
\begin{enumerate}

\item Part numbers to be sorted are unique from part numbers
which are produced from the sort.

\item A part number is sorted once and only once.
 
\end{enumerate}

\noindent
{\bf File Format Rules and Cautions:}
\begin{itemize}
\item see the SCE Data Model for Release 2.0
\end{itemize}

\clearpage
\noindent
{\bf New User Input File Format Specifications for Sorting}
 
\vspace{.5in}

\begin{tabular}{llp{4in}}
\multicolumn{3}{c}{{\bf Sort Distribution File}}\\ \hline\hline
{\bf DataElement} &  {\bf type}  &   {\bf comment} \\ \hline
partNumber1 &  ``char'' &    part number to be sorted 
                             (partNumber1 is of part type 1) \\
partNumber2 &  ``char'' &    part number resulting from sorting
                             partNumber1 (partNumber2 is of part type 1) \\
PDF     &   ``char'' \\
\multicolumn{3}{c} \dotfill \\
distribution\_rate     &   float  &  the percentage of partNumber2s
                                     resulting from sorting
                                     partNumber1, expressed as a
                                     decimal.; default $=$ 0.0 \\
offset    &    float  &     offset from partNumber1's start-to-sort
                             date that partNumber2
                             is produced; default $=$ 0.0 \\
start   &   date  &     date this distribution record becomes active;
                               default $=$ PAST  \\
end     &   date  &     last date this distribution record is active; 
                                default $=$ FUTURE \\

\end{tabular}

\vspace{.5in}

\begin{tabular}{llp{4in}}
\multicolumn{3}{c}{{\bf Sort Yield File}}\\ \hline\hline
{\bf DataElement} &  {\bf type}  &   {\bf comment} \\ \hline
partNumber & ``char'' &  part number to be sorted \\
PDF   & ``char''  \\
yield     &      int  &    yield on sorting partNumber; 1 $<=$ yield $<=$ 100 \\
\multicolumn{3}{c} \dotfill \\
start     &      date &    date this yield data becomes effective; 
        default $=$ PAST \\
end       &      date &    last date this yield data is effective;
        default $=$ FUTURE
\end{tabular}

\vspace{.5in}
\begin{tabular}{llp{4in}}
\multicolumn{3}{c}{{\bf Sort Lot Size File}}\\ \hline\hline
{\bf DataElement} &  {\bf type}  &   {\bf comment} \\ \hline
partNumber & ``char'' &  part number to be sorted \\
PDF   & ``char''  \\
minimum\_lot\_size     &    float  &  minimum lot size for
                            sorting partNumber;
                            0.0 $<=$ minimum\_lot\_size; default
    $=$ 1.0 \\
incremental\_lot\_size     &  float  & incremental lot size for
                             sorting partNumber;
                             1.0 $<=$ incremental\_lot\_size; 
                             default $=$ 1.0\\
\multicolumn{3}{c} \dotfill \\
start     &      date &    date this lot size data becomes effective; 
        default $=$ PAST \\
end       &      date &    last date this lot size data is effective;
        default $=$ FUTURE
\end{tabular}

\vspace{.5in}

\begin{tabular}{llp{4in}}
\multicolumn{3}{c}{{\bf Sort BOC File}}\\ \hline\hline
{\bf DataElement} &  {\bf type}  &   {\bf comment} \\ \hline
partNumber &  ``char'' &    part to be sorted (partNumber is of part type 1) \\
capacityName &  ``char'' &  capacity used in sorting partNumber \\
PDF     &   ``char'' \\
\multicolumn{3}{c} \dotfill \\
usage\_rate     &   float  &    quantity per; default $=$ 1.0 \\
offset    &    float  &    offset from start-to-sort date that the capacity
                             is consumed; default $=$ 0.0 \\
start   &   date  &     date this capacity data becomes active; 
      default $=$ PAST  \\
end     &   date  &     last date this capacity data is active;
      default $=$ FUTURE
\end{tabular}

\vspace{.5in}

\noindent
{\bf Usage Notes:}
\begin{itemize}
\item Any (part number, pdf) entry appearing in the Sort
Distribution File, Sort Yield File, Sort Lot Size File or Sort BOC 
File must be defined in the Part Definition File.
\item Any (capacity name, pdf) entry appearing in the Sort BOC
File must be defined in the Capacity Definition File.
\item The Sort Yield File allows the user to explicity state the yield
when sorting a part number. A Sort Yield File entry
 \begin{center}
\begin{tabular}{lllll}
  ``00000ABC123'', &``970'', &0.9, &start, &end
\end{tabular}
\end{center}
signifies that if 100 units of part 00000ABC123 are sorted, ($100*0.9=$)
90 units will pass and ($100-90=$) 10 units will be lost due yield.

The distributions rates specified in the Sort Distribution File are
applied to the post-yield quantities. Continuing the example, suppose we
also had the following entries in the Sort Distribution File
\begin{center}
\begin{tabular}{llll}
  ``00000ABC123'', &``00000ABC111'', &``907'', &0.1 \\
  ``00000ABC123'', &``00000ABC222'', &``907'', &0.6 \\
  ``00000ABC123'', &``00000ABC333'', &``907'', &0.3
\end{tabular}
\end{center}
Together with the yield, this signifies that if 100 units of part
00000ABC123 are sorted, ($100*0.9*0.1=)$ 9 units of 00000ABC111 will be
produced, ($100*0.9*0.6=$) 54 units of 00000ABC222 will be produced, and
($100*0.9*0.3=$) 27 units 00000ABC333 will be produced. 

Notice that in this example the distribution rates for
part 00000ABC123 sum to 1.0. The distribution rates for a part need
not sum to 1.0. Commonly, the distribution rates sum to less than
1.0, with the remainder being regarded as yield loss.

For example, suppose the Sort Distribution File entries for part
00000XYZ789 were
\begin{center}
\begin{tabular}{llll}
  ``00000XYZ789'', &``00000XYZ111'', &``907'', &0.09 \\
  ``00000XYZ789'', &``00000XYZ222'', &``907'', &0.40 \\
  ``00000XYZ789'', &``00000XYZ333'', &``907'', &0.27
\end{tabular}
\end{center}
The  sum  of the distribution rates for part 00000XYZ789  is
($0.09+0.40+0.27=$) 0.76. If no entries for part 00000XYZ789 are made in
the Sort Yield File  (so, the sort  of  this part assumes the  default
yield  of  1.0)  then  this signifies  that  if  100 units  of   part
00000XYZ789  are  sorted   ($100*1.0*0.09=$)  9 units    of 00000XYZ111,
($100*1.0*0.4$) 40 units of 00000XYZ222, and ($100*1.0*.27=$) 27 units of
00000XYZ333  are produced, with the remainder ($100-9-40-27=22$ units)
attributed to yield loss.

We caution the user againist entering yield information both in the
Sort Distribution File (as distribution rates for a part summing to
less than 1.0) and in the Sort Yield File, as this can lead to
unintentional ``double counting'' of the yield. 

\end{itemize}



\clearpage
%%%%%%%%%%%%%%%%%%%%%%%%%%%%%%%%%%%%%%%%%%%%%%%%%%%%%%%%%%%%%%%%%%%%%%%%%%%%
\noindent
{\bf Generating SCE Release 2.0 Input Files from the Four
(operation-less) Sorting Input Files}
 
\vspace{.5in}

\begin{enumerate}
%%% 1. Sort Distribution File
\item For each record in the Sort Distribution File,
\begin{center}
\begin{tabular}{lllllll}
  ``PN\_1'', &``PN\_2'', &``pdf'', &distr\_rate, &offset,  &start, &end
\end{tabular}
\end{center}
the following steps should be taken:

\begin{enumerate}
\item Concatenate ``PN\_1'' and the string ``\_sortOperation''

%%% Define Operation
\item Add the following record to the Operation Definition File:
\begin{center}
\begin{tabular}{ll}
 ``PN\_1\_sortOperation'', &``pdf''
\end{tabular}
\end{center}

%%% Define the Bom's
\item Add the following record to the Manufacturing BOM File
\begin{center}
\begin{tabular}{lllllll}
 ``PN\_1\_sortOperation'', &``PN\_1'', &``pdf'', &1.0, &0.0, &start, &end
\end{tabular}
\end{center}

%%% Define the BOP's
\item Add the following record to the BOP Definition File:
\begin{center}
\begin{tabular}{lllllll}
 ``PN\_1'', &``PN\_1\_sortOperation'',  &``pdf'', &distr\_rate, &offset, &start, &end
\end{tabular}
\end{center}

\end{enumerate}
\vspace{.5in}

%%% 2. Sort BOC File
\item For each record in the Sort BOC File,
\begin{center}
\begin{tabular}{lllllll}
  ``PN\_1'', &``capacityName'', &``pdf'', &usage\_rate, &offset,  &start, &end
\end{tabular}
\end{center}
the following steps should be taken:

\begin{enumerate}
\item Concatenate ``PN\_1'' and the string ``\_sortOperation''

\item Add the following record to the Manufacturing BOC File:
\begin{center}
\begin{tabular}{lllllll}
 ``PN\_1\_sortOperation'', &``capacityName'', &``pdf'', &usage\_rate, &offset,  &start, &end
\end{tabular}
\end{center}

\end{enumerate}
\vspace{.5in}

%%% 3. Sort Lot Size File
\item For each record in the Sort Lot Size File,
\begin{center}
\begin{tabular}{llllll}
  ``PN\_1'', &``pdf'', &min\_lotsize, &inc\_lotsize, &start, &end
\end{tabular}
\end{center}
the following steps should be taken:

\begin{enumerate}
\item Concatenate ``PN\_1'' and the string ``\_sortOperation''

\item Add the following record to the Operation Lot Size File:
\begin{center}
\begin{tabular}{llllll}
 ``PN\_1\_sortOperation'', &``pdf'', &min\_lotsize, &inc\_lotsize, &start, &end 
\end{tabular}
\end{center}

\end{enumerate}
\vspace{.5in}

%%% 4. Sort Yield File
\item For each record in the Sort Yield File,
\begin{center}
\begin{tabular}{lllll}
  ``PN\_1'', &``pdf'', &yield, &start, &end
\end{tabular}
\end{center}
the following steps should be taken:

\begin{enumerate}
\item Concatenate ``PN\_1'' and the string ``\_sortOperation''

\item Add the following record to the Operation Yield Size File:
\begin{center}
\begin{tabular}{lllll}
 ``PN\_1\_sortOperation'', &``pdf'', &yield, &start, &end
\end{tabular}
\end{center}
\end{enumerate}

\end{enumerate}
\vspace{.5in}


%%%%%%%%%%%%%%%%%%%%%%%%%%%%%%%%%%%%%%%%%%%%%%%%%%%%%%%%

\end{document}
